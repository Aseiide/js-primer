\hypertarget{use-case}{%
\part{ユースケース}\label{use-case}}

第一部の基本文法で学んだことを応用し、具体的なユースケースを元に学んでいきます。

\hypertarget{summary}{%
\section*{目次}\label{summary}}

\hypertarget{setup-local-env}{%
\subsection*{\texorpdfstring{\href{./setup-local-env/README.md}{アプリケーション開発の準備}}{アプリケーション開発の準備}}\label{setup-local-env}}

アプリケーション開発のためにNode.jsとnpmのインストールなどの準備方法を紹介します。

\hypertarget{ajaxapp}{%
\subsection*{\texorpdfstring{\href{./ajaxapp/README.md}{Ajaxで通信}}{Ajaxで通信}}\label{ajaxapp}}

ウェブブラウザ上でAjax通信をするユースケースとして、GitHubのユーザーIDからプロフィール情報を取得するアプリケーションを作成しながら、非同期処理について紹介します。

\hypertarget{nodecli}{%
\subsection*{\texorpdfstring{\href{./nodecli/README.md}{Node.jsでCLIアプリ}}{Node.jsでCLIアプリ}}\label{nodecli}}

Node.jsでCLI(コマンドラインインターフェース)アプリケーションを開発する例として、MarkdownをHTMLに変換するツールを作成していきます。また、Node.jsやnpmの使い方を紹介します。

\hypertarget{todoapp}{%
\subsection*{\texorpdfstring{\href{./todoapp/README.md}{Todoアプリ}}{Todoアプリ}}\label{todoapp}}

ブラウザで動作するウェブアプリケーションの例としてTodoアプリを作成しながら、モジュールを使ったコード管理について紹介します。
